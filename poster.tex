\documentclass{beamer}
% 使用LuaLaTeX编译时不需要fontenc
\usepackage{fontspec} % 用于OpenType字体管理
\setsansfont{TeX Gyre Heros} % 类似Helvetica的无衬线字体
\setmainfont{TeX Gyre Termes} % 类似Times的衬线字体

% 保持其他包不变,但移除fontenc和传统字体包
\usepackage[size=custom, width=91.44, height=60.94, scale=1.0]{beamerposter}
\usepackage[UTF8]{ctex}

% 使用unicode-math设置数学字体
\usepackage{unicode-math}
\setmathfont{TeX Gyre Termes Math} % 与主文字体匹配的数学字体

% Theme and color theme
\usetheme{bnu}
\usecolortheme{bnucolor}

% 其他包保持不变
\usepackage{graphicx,booktabs,tikz,pgfplots,amsfonts,amsmath,amssymb,amsthm,hyperref,caption}

% Tikz Customization
\usetikzlibrary{intersections}
\usepgfplotslibrary{fillbetween}

% PGFPlots config
\pgfplotsset{compat=1.18} % 更新到较新版本

% Dimensions for column layout
\newlength{\sepwidth}
\newlength{\colwidth}
\setlength{\sepwidth}{0.009\paperwidth}
\setlength{\colwidth}{0.3\paperwidth}

\newcommand{\separatorcolumn}{\begin{column}{\sepwidth}\end{column}}

% Title and authors
\title{Mathematical Foundations }
\author{Issac Newton \inst{1} \and Alan Turing \inst{1} \and Leonard Euler \inst{2}}
\institute[shortinst]{\inst{1} University of Cambridge \qquad \inst{2} University of Basel}

% Footer content
\footercontent{
\href{https://github.com/mrislambd/bnu-math-poster-template}{github.com/mrislambd/bnu-math-poster-template} \hfill Unofficial Poster Design\hfill Beijing Normal University Mathematics
}

% Logos (replace with your real paths)
\logoleft{\includegraphics[height=6.5cm]{logo/bnu-logo.png}}
\logoright{\includegraphics[height=7cm]{logo/bnuphys.png}}

\begin{document}

\begin{frame}[t]
\begin{columns}[t]

% COLUMN 1
\begin{column}{\colwidth}
\begin{block}{介绍}
这是一个模板,This is an unofficial template for Florida State Mathematics poster presentation prepared by Rafiq Islam\cite{bnumathposter25}. Here is how you use plain text\hfill\\
Spectral graph theory studies properties of a graph in relationship to the eigenvalues and eigenvectors of matrices associated with the graph, such as the adjacency matrix $A$, degree matrix $D$, and Laplacian $L = D - A$.
\end{block}

\begin{block}{Mathematical Background}
    Let $G = (V, E)$ be an undirected graph. The Laplacian matrix is given by
        \[
        L_{ij} =
        \begin{cases}
        \deg(v_i) & \text{if } i = j, \\
        -1 & \text{if } i \neq j \text{ and } (i,j) \in E, \\
        0 & \text{otherwise}.
        \end{cases}
        \]

        The eigenvalues of $L$ reveal key structural properties such as connectivity.
\end{block}

\begin{block}{Tikz Picture}
    \begin{figure}
        \centering
        \begin{tikzpicture}[scale=4]
            \draw[thick] (0,0) ellipse (1.2 and 0.4);
            \draw[thick] (0,-2) ellipse (1.2 and 0.4);
            \draw[thick] (-1.2,0) -- (-1.2,-2);
            \draw[thick] (1.2,0) -- (1.2,-2);
            \node at (0, -2.5) {radius $r = 5$};
            \node[right] at (1.3,-1) {height $h = 10$};
        \end{tikzpicture}
        \caption{Spherical Cylinder with radius $r=5$m and height $h=10$ m}
        \label{fig:fig1}
    \end{figure}
\end{block}

\begin{block}{Objectives}
\begin{itemize}
  \item Explore classical ML models for text classification
  \item Apply preprocessing techniques like TF-IDF
  \item Evaluate models using accuracy, precision, recall
\end{itemize}
\end{block}

\begin{alertblock}{Why Text Classification?}
This is an alert block where you can mention an important fact or result.
\begin{itemize}
  \item Used in spam detection, sentiment analysis, topic labeling
  \item Helps automate information filtering at scale
\end{itemize}
\end{alertblock}
\end{column}

\separatorcolumn

% COLUMN 2
\begin{column}{\colwidth}
\begin{block}{Methodology}
Numbered items where the number elements will display in garnet color.\hfill
\begin{enumerate}
  \item Dataset cleaning: lowercasing, punctuation removal
  \item Tokenization and stop-word removal
  \item Feature extraction via TF-IDF
  \item Training Logistic Regression, Naive Bayes, and SVM
\end{enumerate}
\end{block}

\begin{block}{Visualization}
\begin{figure}
    \centering
    \includegraphics[width=0.5\linewidth]{images/seminole.png}
    \caption{Florida State Seminole (Photo credit: \href{https://en.wikipedia.org/wiki/Florida_State_Seminoles}{Wikipedia})}
    \label{fig:enter-label}
\end{figure}
\end{block}

\begin{block}{Table}
    \begin{center}
        \renewcommand{\arraystretch}{1.4}
        \begin{tabular}{|c|c|c|c|}
        \hline
        \textbf{Graph} & \textbf{Nodes ($n$)} & \textbf{Edges ($m$)} & \textbf{2nd Eigenvalue $\lambda_2$} \\
        \hline
        Cycle $C_6$ & 6 & 6 & 1.0 \\
        Complete $K_4$ & 4 & 6 & 4.0 \\
        Star $S_5$ & 5 & 4 & 1.0 \\
        Path $P_5$ & 5 & 4 & 0.3819 \\
        Random $G_{10,0.5}$ & 10 & 23 & 1.823 \\
        \hline
        \end{tabular}
        \vspace{0.5em}
        \captionof{table}{Comparison of different graphs and their spectral properties.}
    \end{center}
\end{block}

\end{column}


\separatorcolumn

% COLUMN 3
\begin{column}{\colwidth}
\begin{exampleblock}{More General block}
    The more accurate approximation could be found by increasing the number of sub-intervals. That is
    \begin{equation}
        \int_0^{20} f(x)dx = \lim_{n \to \infty} \sum_{i=1}^n f(x_i^*)\Delta x\approx 144.15
    \end{equation}    
    irrespective of the left-hand point or the right-hand point we take to draw the rectangles.
        \begin{center}
        \begin{tikzpicture}
            \begin{axis}[
                width=14cm,
                height=7cm,
                xmin=-1.5, xmax=22,
                ymin=-20, ymax=30,
                axis lines=middle,
                xlabel={$x$},
                ylabel={$f(x)$},
                samples=300,
                domain=0:22,
                grid=both,
            ]
            
            % Area under the curve
            \addplot [domain=-1.5:22, samples=300, name path=f, thick, color=blue]
            {1+0.25*x+sin(deg(x)) + 20*sin(deg(x)/2)};

            \path [name path=xaxis]
            (\pgfkeysvalueof{/pgfplots/xmin},0) --
            (\pgfkeysvalueof{/pgfplots/xmax},0);

            \addplot[blue!50, opacity=0.4] fill between [of=f and xaxis, soft clip={domain=-0:20}];
            
            \end{axis}
        \end{tikzpicture}
        \end{center}
        This expression
        \begin{equation*}
            \int_0^{20} f(x)dx   
        \end{equation*}
        is called the definite integral that finds the area under and/or above the curve $f(x)$ and the $x$ axis.
\end{exampleblock}

\begin{exampleblock}{Conclusion}
\begin{itemize}
  \item SVM performed best with highest accuracy
  \item TF-IDF improved performance over raw counts
  \item Logistic Regression was more interpretable
\end{itemize}
\end{exampleblock}

\begin{block}{References}
\small
\footnotesize{\bibliographystyle{plain}\bibliography{ref}}
\end{block}
\end{column}

\end{columns}
\end{frame}
\end{document}
