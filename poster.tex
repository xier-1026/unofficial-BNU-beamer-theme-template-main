\documentclass{beamer}
\usepackage{fontspec}
\setsansfont{TeX Gyre Heros}
\setmainfont{TeX Gyre Termes}

\usepackage[size=custom, width=91.44, height=60.94, scale=1.0]{beamerposter}
\usepackage[UTF8]{ctex}

\usepackage{unicode-math}
\setmathfont{TeX Gyre Termes Math}

\usetheme{bnu}
\usepackage{fontspec}
\setsansfont{TeX Gyre Heros}
\setmainfont{TeX Gyre Termes}

\usepackage[size=custom, width=91.44, height=121.92, scale=1.0]{beamerposter}
\usepackage[UTF8]{ctex}

\usepackage{unicode-math}
\setmathfont{TeX Gyre Termes Math}

\usetheme{bnu}
\usecolortheme{bnucolor}

\usepackage{graphicx,booktabs,tikz,pgfplots,amsfonts,amsmath,amssymb,amsthm,hyperref,caption}

\usetikzlibrary{intersections}
\usepgfplotslibrary{fillbetween}

\pgfplotsset{compat=1.18}

\newlength{\sepwidth}
\newlength{\colwidth}
\setlength{\sepwidth}{0.009\paperwidth}
\setlength{\colwidth}{0.3\paperwidth}

\newcommand{\separatorcolumn}{\begin{column}{\sepwidth}\end{column}}

% Title and authors
\title{乌鸦悖论——系统科学哲学的审视与超越}
\author{Xi'er=1026}
\institute[shortinst]{School of Physics and Astronomy,Beijing Normal University}

% Footer content
\footercontent{
\href{https://github.com/xier-1026/unofficial-BNU-beamer-theme-template-main} \hfill 学术海报设计\hfill 北京师范大学
}

\logoleft{\includegraphics[height=6.5cm]{logo/bnu-logo-white.png}}
\logoright{\includegraphics[height=7cm]{logo/bnuphys.png}}

\begin{document}

\begin{frame}[t]
\begin{columns}[t]

% COLUMN 1
\begin{column}{\colwidth}
\begin{block}{引言:一个令理性尴尬的难题}
\begin{columns}
\begin{column}{0.6\textwidth}
\begin{itemize}
    \item \textbf{逻辑铁律}:"所有乌鸦都是黑的" $\Leftrightarrow$ "所有非黑的都不是乌鸦"
    \item \textbf{悖论现场}:观察一支\textbf{白色粉笔},在逻辑上确证了乌鸦的颜色
    \item \textbf{核心冲突}:\textbf{无懈可击的逻辑} vs. \textbf{强烈排斥的直觉}
\end{itemize}
\end{column}
\begin{column}{0.4\textwidth}
\includegraphics[width=0.6\textwidth]{images/image-41.png}
\end{column}
\end{columns}
\end{block}

\begin{block}{贝叶斯方法:技术性修复}
\begin{itemize}
    \item \textbf{方案}:引入概率计算,白粉笔的"确证度"极低
    \item \textbf{贡献}:在\textbf{数学上}巧妙地"解决"了问题
    \item \textbf{局限}:它回答了 \textbf{"多少"},但未回答 \textbf{"为何"}
    \begin{itemize}
        \item 为何我们的认知系统会天然地排斥这个逻辑?
    \end{itemize}
\end{itemize}
\end{block}

\begin{block}{系统科学视角:范式的转换}
\begin{columns}
\begin{column}{0.6\textwidth}
\begin{itemize}
    \item \textbf{历史背景}:冷战工程催生"三论"
    \begin{itemize}
        \item 系统论、控制论、信息论
    \end{itemize}
    \item \textbf{核心思想}:世界是\textbf{整体的、层次的、动态的}\cite{BNUbeamerpresent00}
    \item \textbf{我们的视角}:"所有乌鸦都是黑的"是一个\textbf{活系统}中的命题
\end{itemize}
\end{column}
\begin{column}{0.4\textwidth}
\includegraphics[width=0.9\textwidth]{images/image-42.png}
\end{column}
\end{columns}
\end{block}

\begin{block}{系统科学理论基础}
\begin{columns}
\begin{column}{0.5\textwidth}
\begin{itemize}
    \item \textbf{八原理}:描述系统的\textbf{静态存在方式与基本属性}
    \item \textbf{五规律}:揭示系统的\textbf{动态演化过程与内在机制}
\end{itemize}
\end{column}
\begin{column}{0.5\textwidth}
\includegraphics[width=0.9\textwidth]{images/image-45.png}
\end{column}
\end{columns}
\end{block}
\end{column}

\separatorcolumn

% COLUMN 2
\begin{column}{\colwidth}
\begin{block}{悖论源于"层次混淆"}
\begin{columns}
\begin{column}{0.5\textwidth}
\begin{itemize}
    \item \textbf{高层-命题层}:假说本身
    \item \textbf{中层-关系层}:
    \begin{itemize}
        \item 直接关系 vs. 间接关系
    \end{itemize}
    \item \textbf{底层-实例层}:
    \begin{itemize}
        \item 高价值实例 vs. 低价值实例
    \end{itemize}
\end{itemize}
\end{column}
\begin{column}{0.5\textwidth}
\includegraphics[width=0.9\textwidth]{images/image-46.png}
\end{column}
\end{columns}
\end{block}

\begin{block}{系统论诊断支持}
\begin{itemize}
    \item \textbf{整体性原理}:命题的意义由整个知识系统赋予
    \item \textbf{结构功能相关律}:证据在系统结构中的\textbf{位置},决定了其确证\textbf{功能}
    \item \textbf{竞争协同律}:逻辑与直觉的冲突,正是系统演化的内在动力
\end{itemize}
\end{block}

\begin{alertblock}{三重认知陷阱}
\begin{block}{A (有价值的核心命题)}
建立可靠的科学确证理论
\end{block}

\begin{block}{B (有缺陷的实现方式)}
对"逻辑等价原则"的僵化应用
\end{block}

\begin{itemize}
    \item 关键在于区分A和B
    \item 坚持A,批判B
\end{itemize}
\end{alertblock}

\begin{block}{认知陷阱分析}
\begin{columns}[t]
\begin{column}{0.48\textwidth}
\begin{itemize}
    \item \textbf{陷阱一:顺从陷阱}
    \begin{itemize}
        \item 承认"白粉笔确证乌鸦颜色"
        \item 为逻辑牺牲直觉
        \item 理性被异化
    \end{itemize}
\end{itemize}
\end{column}

\begin{column}{0.48\textwidth}
\begin{itemize}
    \item \textbf{陷阱二:反对陷阱}
    \begin{itemize}
        \item 因结论荒谬怀疑科学确证理论
        \item 动摇科学根基
        \item 滑向虚无主义
    \end{itemize}
\end{itemize}
\end{column}
\end{columns}
\end{block}
\end{column}

\separatorcolumn

% COLUMN 3
\begin{column}{\colwidth}
\begin{block}{解决方案:系统思维的胜利}
\begin{columns}
\begin{column}{0.6\textwidth}
\begin{itemize}
    \item \textbf{核心操作}:坚定A,批判B,重构系统
    \item 悖论作为\textbf{"涨落"},触发认知系统的\textbf{"自组织"}
    \item 通过\textbf{"反馈"}走向更高级的有序
    \item 如贝叶斯模型的引入
\end{itemize}

\begin{quote}
真正的出路是'站着'的。我们忠于科学的目标,但敢于改造科学的工具。
\end{quote}
\end{column}
\begin{column}{0.4\textwidth}
\includegraphics[width=0.4\textwidth]{images/image-43.png}
\end{column}
\end{columns}
\end{block}

\begin{block}{机制的普遍性}
\begin{columns}[t]
\begin{column}{0.48\textwidth}
\begin{itemize}
    \item \textbf{A:正当的核心命题}
    \begin{itemize}
        \item 平权
        \item 知识  
        \item 爱国
        \item 科学精神
    \end{itemize}
\end{itemize}
\end{column}

\begin{column}{0.48\textwidth}
\begin{itemize}
    \item \textbf{B:有缺陷的实现方式}
    \begin{itemize}
        \item 政治正确
        \item 应试教育
        \item 狭隘民族主义
        \item 教条主义科学观
    \end{itemize}
\end{itemize}
\end{column}
\end{columns}

\begin{center}
\begin{minipage}{0.9\textwidth}
\begin{itemize}
    \item \textbf{核心启示}:坚持崇高的核心价值(A),同时批判和改进有缺陷的实现方式(B)
\end{itemize}
\end{minipage}
\end{center}
\end{block}

\begin{exampleblock}{结论与启示}
\begin{itemize}
    \item \textbf{世界观}:世界是相互关联的复杂系统
    \item \textbf{认识论}:知识具有层级结构,警惕范畴错误
    \item \textbf{方法论}:动态、反馈、自组织的分析框架
    \item \textbf{核心价值}:系统思维提供理解复杂性的有效工具
\end{itemize}
\end{exampleblock}

\begin{block}{讨论:悖论解决了吗?}
\begin{columns}[t]
\begin{column}{0.48\textwidth}
\begin{itemize}
    \item \textbf{A:没解决}
    \begin{itemize}
        \item 解决的同时坐实了问题的复杂性
        \item 表明需要\textbf{更丰富模型}才能理解
        \item 从\textbf{逻辑谜题}升级为\textbf{系统性问题}
    \end{itemize}
\end{itemize}
\end{column}

\begin{column}{0.48\textwidth}
\begin{itemize}
    \item \textbf{B:解决了}
    \begin{itemize}
        \item 复杂化是将问题\textbf{引回现实语境}
        \item 提供\textbf{语法+语义+语用}的完整分析
        \item 哲学服务于\textbf{认识现实、改造现实}的需求
    \end{itemize}
\end{itemize}
\end{column}
\end{columns}
\end{block}

\begin{block}{参考文献}
\small
\footnotesize{\bibliographystyle{plain}\bibliography{ref}}
\end{block}
\end{column}

\end{columns}
\end{frame}
\end{document}
