\documentclass{beamer}
\usepackage{fontspec} % LuaLaTeX字体管理
\setsansfont{TeX Gyre Heros} % 类似Helvetica的无衬线字体
\setmainfont{TeX Gyre Termes} % 类似Times的衬线字体

\usepackage[UTF8]{ctex}

% 使用unicode-math设置数学字体
\usepackage{unicode-math}
\setmathfont{TeX Gyre Termes Math}

% This part is required to have bnu custom style
\usetheme[presentation]{bnu}
\usecolortheme[presentation]{bnucolor}

\AtBeginSection{} % 禁用章节开始的自动目录
\AtBeginSubsection{} % 禁用于章节开始的自动目录

% Include the required packages here
\usepackage{graphicx,booktabs,tikz,pgfplots,amsfonts,amsmath,amsthm,hyperref,caption,amssymb}

% Tikz Customization
\usetikzlibrary{intersections}
\usepgfplotslibrary{fillbetween}

% PGFPlots config
\pgfplotsset{compat=1.18}

% 使用fontspec设置字体,移除newtxtext和newtxmath
\usefonttheme{serif} % 使用衬线字体(包括数学公式)

% Logo in the first page, comment out if don't wanted
\presentationlogo{\includegraphics[width=4cm]{logo/bnu.png}}
%%%%%%%%%%%%%%%%%%%%%%%%%%%%%%%%%%%%%%%%%%%%%%%%%%%%%%%
\title[在意识形态陷阱与科学哲学之间]{乌鸦悖论——系统科学哲学的审视与超越}
\author[刘子旗]{刘子旗}
\institute{物理与天文学院 光学}
\date{2025年11月13日}

\begin{document}

% title page
\maketitle

% Table of contents
\begin{frame}{Table of Contents}
\tableofcontents[subsectionstyle=hide, sectionstyle=show]
\end{frame}
% Intro section
\section{引言}
% A subsection. May contain multiple subsections
\subsection{乌鸦悖论}
\begin{frame}{一个令理性尴尬的难题}
\begin{columns}
\begin{column}{0.6\textwidth}
\begin{itemize}
    \item \textbf{逻辑铁律}:"所有乌鸦都是黑的" $\Leftrightarrow$ "所有非黑的都不是乌鸦"
    \item \textbf{悖论现场}:观察一支\textbf{白色粉笔},在逻辑上确证了乌鸦的颜色
    \item \textbf{核心冲突}:\textbf{无懈可击的逻辑} vs. \textbf{强烈排斥的直觉}
\end{itemize}
\end{column}
\begin{column}{0.4\textwidth}
\includegraphics[width=1.0\textwidth]{images/image-41.png}
\end{column}
\end{columns}
\end{frame}

\subsection{贝叶斯方法}
\begin{frame}{贝叶斯的确证度:一次技术性修复}
\begin{itemize}
    \item \textbf{方案}:引入概率计算,白粉笔的"确证度"极低
    \item \textbf{贡献}:在\textbf{数学上}巧妙地"解决"了问题
    \item \textbf{局限}:它回答了 \textbf{"多少"},但未回答 \textbf{"为何"}
    \begin{itemize}
        \item 为何我们的认知系统会天然地排斥这个逻辑?
    \end{itemize}
\end{itemize}
\end{frame}
%%%%%%%%%%%%%%%%%%%%%%%%%%%%%%%%%%%
\section{系统科学视角}
% A subsection. May contain multiple subsections
\subsection{系统论}
\begin{frame}{范式的转换:从“孤立命题”到“知识系统”}
\begin{columns}
\begin{column}{0.6\textwidth}
\begin{itemize}
    \item \textbf{历史背景}:冷战工程催生"三论"
    \begin{itemize}
        \item 系统论、控制论、信息论
    \end{itemize}
    \item \textbf{核心思想}:世界是\textbf{整体的、层次的、动态的}\cite{BNUbeamerpresent00}
    \item \textbf{我们的视角}:"所有乌鸦都是黑的"是一个\textbf{活系统}中的命题
\end{itemize}
\end{column}
\begin{column}{0.4\textwidth}
\includegraphics[width=0.9\textwidth]{images/image-42.png}
\end{column}
\end{columns}
\end{frame}

\subsection{系统论基本内容}
\begin{frame}{系统科学的理论基石}
\begin{columns}
\begin{column}{0.5\textwidth}
\begin{block}{是什么:八原理}
主要描述系统的\textbf{静态存在方式与基本属性}
\end{block}

\vspace{0.5cm}

\begin{block}{如何变:五规律}
主要揭示系统的\textbf{动态演化过程与内在机制}
\end{block}
\end{column}
\begin{column}{0.5\textwidth}
\includegraphics[width=0.9\textwidth]{images/image-45.png}
\end{column}
\end{columns}
\end{frame}
%%%%%%%%%%%%%%%%%%%%%%%%%%%%%%%%%%%%%%
\begin{frame}{悖论源于"层次混淆"}
\begin{columns}
\begin{column}{0.5\textwidth}
\begin{itemize}
    \item \textbf{高层-命题层}:假说本身
    \item \textbf{中层-关系层}:
    \begin{itemize}
        \item 直接关系 vs. 间接关系
    \end{itemize}
    \item \textbf{底层-实例层}:
    \begin{itemize}
        \item 高价值实例 vs. 低价值实例
    \end{itemize}
\end{itemize}
\end{column}
\begin{column}{0.5\textwidth}
\includegraphics[width=0.9\textwidth]{images/image-46.png}
\end{column}
\end{columns}
\end{frame}

\begin{frame}{系统论如何支持这一诊断?}
\begin{itemize}
    \item \textbf{整体性原理}:命题的意义由整个知识系统赋予
    \item \textbf{结构功能相关律}:证据在系统结构中的\textbf{位置},决定了其确证\textbf{功能}
    \item \textbf{竞争协同律}:逻辑与直觉的冲突,正是系统演化的内在动力
\end{itemize}
\end{frame}

% 认知陷阱部分
\subsection{认知陷阱分析}
\begin{frame}{思维如何被俘获?三重认知陷阱}
\begin{block}{A (有价值的核心命题)}
建立可靠的科学确证理论
\end{block}

\begin{block}{B (有缺陷的实现方式)}
对"逻辑等价原则"的僵化应用
\end{block}

\begin{itemize}
    \item 关键在于区分A和B
    \item 坚持A,批判B
\end{itemize}
\end{frame}

\begin{frame}{认知陷阱:顺从 vs 反对}
\begin{columns}[t] % [t] 参数确保两列顶部对齐
\begin{column}{0.52\textwidth} % 稍微小于0.5以便留出间距
\begin{block}{陷阱一:顺从陷阱}
\begin{itemize}
    \item \textbf{表现}:承认"白粉笔确证乌鸦颜色",为逻辑牺牲直觉
    \item \textbf{结果}:理性被异化,科学与常识脱节
    \item \textbf{根源}:
    \begin{itemize}
        \item 过度依赖形式逻辑
        \item 忽视认知系统的整体性
    \end{itemize}
\end{itemize}
\end{block}
\end{column}

\begin{column}{0.52\textwidth}
\begin{block}{陷阱二:反对陷阱}
\begin{itemize}
    \item \textbf{表现}:因结论荒谬,转而怀疑整个科学确证理论
    \item \textbf{结果}:动摇科学根基,滑向虚无主义
    \item \textbf{根源}:
    \begin{itemize}
        \item 全盘否定科学方法
        \item 忽视逻辑的有效性
    \end{itemize}
\end{itemize}
\begin{quote}
\small 倒洗澡水时连孩子一起倒掉了
\end{quote}
\end{block}
\end{column}
\end{columns}
\end{frame}

% 解决方案部分
\section{解决方案}
\begin{frame}{出路:超越·系统思维的胜利}
\begin{columns}
\begin{column}{0.6\textwidth}
\begin{block}{核心操作}
\textbf{坚定A,批判B,重构系统}
\end{block}

\begin{itemize}
    \item 悖论作为\textbf{"涨落"},触发认知系统的\textbf{"自组织"}
    \item 通过\textbf{"反馈"}走向更高级的有序
    \item 如贝叶斯模型的引入
\end{itemize}

\begin{quote}
真正的出路是'站着'的。我们忠于科学的目标,但敢于改造科学的工具。
\end{quote}
\end{column}
\begin{column}{0.4\textwidth}
\includegraphics[width=0.7\textwidth]{images/image-43.png}
\end{column}
\end{columns}
\end{frame}

\begin{frame}{机制的普遍性}
\begin{columns}[t] % [t] 参数确保两栏顶部对齐
\begin{column}{0.48\textwidth}
\begin{block}{A:正当的核心命题}
\begin{itemize}
    \item 平权
    \item 知识  
    \item 爱国
    \item 科学精神
\end{itemize}
\end{block}
\end{column}

\begin{column}{0.48\textwidth}
\begin{block}{B:有缺陷的实现方式}
\begin{itemize}
    \item 政治正确的
    \item 应试教育
    \item 狭隘民族主义
    \item 教条主义科学观
\end{itemize}
\end{block}
\end{column}
\end{columns}

\vspace{0.1cm}

\begin{center}
\begin{minipage}{0.9\textwidth}
\begin{block}{核心启示}
\textbf{关键在于区分A与B}:坚持崇高的核心价值(A),同时批判和改进有缺陷的实现方式(B)
\end{block}
\end{minipage}
\end{center}
\end{frame}

% 结论部分
\subsection{结论与启示}
\begin{frame}{从乌鸦到世界:系统思维的普遍启示}

\begin{itemize}
    \item \textbf{世界观}:世界是相互关联的复杂系统
    \item \textbf{认识论}:知识具有层级结构,警惕范畴错误
    \item \textbf{方法论}:动态、反馈、自组织的分析框架
\end{itemize}

\vspace{0.3cm}

\begin{block}{核心价值}
系统思维提供理解复杂性的有效工具
\end{block}
\end{frame}

\begin{frame}{讨论:悖论,是理性前进的路标}
\begin{itemize}
    \item 乌鸦悖论揭示了\textbf{静态逻辑}在处理\textbf{动态认知系统}时的固有局限
    \item 系统科学哲学提供了一套\textbf{元语言},让我们能描述并超越这种局限
    \item 真正的智慧在于:
    \begin{itemize}
        \item 对终极价值(A)的坚守
        \item 对实现路径(B)永不停息的革新
    \end{itemize}
\end{itemize}
\end{frame}

\begin{frame}{讨论:悖论解决了吗?}
\begin{columns}[t]
\begin{column}{0.48\textwidth}
\begin{block}{A:没解决}
\begin{itemize}
    \item 解决的同时坐实了问题的复杂性
    \item 表明需要\textbf{更丰富模型}才能理解
    \item 从\textbf{逻辑谜题}升级为\textbf{系统性问题}
\end{itemize}
\end{block}
\end{column}

\begin{column}{0.48\textwidth}
\begin{block}{B:解决了}
\begin{itemize}
    \item 复杂化是将问题\textbf{引回现实语境}
    \item 提供\textbf{语法+语义+语用}的完整分析
    \item 哲学服务于\textbf{认识现实、改造现实}的需求
\end{itemize}
\end{block}
\end{column}
\end{columns}

\vspace{0.01cm}

\begin{center}
\begin{minipage}{1.0\textwidth}
\begin{block}{警示牌}
在一个复杂、分层、信息丰富的世界中,理性地建立信念
\end{block}
\end{minipage}
\end{center}
\end{frame}

\section{Reference}
\begin{frame}{Bibliography}
\bibliographystyle{plainnat}\bibliography{ref}
\end{frame}

\begin{frame}
\begin{columns}
\begin{column}{0.5\textwidth}
\centering
\vspace{1.5cm}
{\Huge 谢谢聆听}\\
\vspace{1cm}
{\large 刘子旗}\\
{\small 北京师范大学物理与天文学院}\\
{\small 2025年11月13日}
\end{column}
\begin{column}{0.5\textwidth}
\centering
\includegraphics[width=0.5\textwidth]{images/seminole.png}
\end{column}
\end{columns}
\end{frame}

\end{document}