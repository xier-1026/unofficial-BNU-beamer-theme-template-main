\documentclass{beamer}
\usepackage{fontspec} % LuaLaTeX字体管理
\setsansfont{TeX Gyre Heros} % 类似Helvetica的无衬线字体
\setmainfont{TeX Gyre Termes} % 类似Times的衬线字体

\usepackage[UTF8]{ctex}

% 使用unicode-math设置数学字体
\usepackage{unicode-math}
\setmathfont{TeX Gyre Termes Math}

% This part is required to have bnu custom style
\usetheme[presentation]{bnu}
\usecolortheme[presentation]{bnucolor}

% 移除传统字体包
% 使用fontspec和unicode-math替代

% Include the required packages here
\usepackage{graphicx,booktabs,tikz,pgfplots,amsfonts,amsmath,amsthm,hyperref,caption,amssymb}

% Tikz Customization
\usetikzlibrary{intersections}
\usepgfplotslibrary{fillbetween}

% PGFPlots config
\pgfplotsset{compat=1.18}

% 使用fontspec设置字体,移除newtxtext和newtxmath
\usefonttheme{serif} % 使用衬线字体(包括数学公式)

% Logo in the first page, comment out if don't wanted
\presentationlogo{\includegraphics[width=2cm]{logo/bnu.png}}
%%%%%%%%%%%%%%%%%%%%%%%%%%%%%%%%%%%%%%%%%%%%%%%%%%%%%%%
\title[BNU 模板]{BNU beamer 模板示例}
\author[Liu]{Liu Ziqi (刘子旗)}
\institute{Beijing Normal University}
\date{\today}

\begin{document}

% title page
\maketitle

% Table of contents
\begin{frame}{Table of Contents}
\tableofcontents[subsectionstyle=hide, sectionstyle=show]
\end{frame}
% Intro section
\section{介绍}
% A subsection. May contain multiple subsections
\subsection{Graph Theory}
\begin{frame}{What is a graph?}
    This is an unofficial template for Florida State Mathematics poster presentation prepared by Rafiq 
    Islam\cite{bnumathposter25}. Here is how you use plain text. Here is how you can use a block to write some important information \cite{bnubeamerpresent25}
    \begin{block}{Graph Theory}
        Spectral graph theory studies properties of a graph in relationship to the eigenvalues and eigenvectors of matrices associated with the graph, such as the adjacency matrix $A$, degree matrix $D$, and Laplacian $L = D - A$.
    \end{block}
   \begin{itemize}
    \item This is how you can start \texttt{itemize}
    \item Instead of this right-pointed arrow, if you want bullets, then see the instruction in line 6
\end{itemize}
\end{frame}
%%%%%%%%%%%%%%%%%%%%%%%%%%%%%%%%%%%%%%%%%%%%%%%%%%
\subsection{Why Graph Theory}
\begin{frame}{Why do we need this?}
This is an unofficial template for the Florida State Mathematics poster presentation prepared by Rafiq Islam\footnote{An example of footnote}. Here is how you use plain text. Spectral graph theory studies properties of a graph in relationship to the eigenvalues and eigenvectors of matrices associated with the graph, such as the adjacency matrix $A$, degree matrix $D$, and Laplacian $L = D - A$.
\end{frame}
%%%%%%%%%%%%%%%%%%%%%%%%%%%%%%%%%%%
\section{Methodology}
\begin{frame}{Mathematical Background}
    This section has three slides. So in the top right, we see 3 dots. Highlighted one indicates the current slide.\\
    Let $G = (V, E)$ be an undirected graph. The Laplacian matrix is given by
        \[
        L_{ij} =
        \begin{cases}
        \deg(v_i) & \text{if } i = j, \\
        -1 & \text{if } i \neq j \text{ and } (i,j) \in E, \\
        0 & \text{otherwise}.
        \end{cases}
        \]
    \begin{enumerate}
        \item The eigenvalues of $L$ reveal key structural properties such as connectivity.
    \end{enumerate}
\end{frame}
%%%%%%%%%%%%%%%%%%%%%%%%%%%%%%%%%%%%%%%%%%%
\begin{frame}{A Tikz Picture Example}
    \begin{figure}
        \centering
        \begin{tikzpicture}
            \draw[thick] (0,0) ellipse (1.2 and 0.4);
            \draw[thick] (0,-2) ellipse (1.2 and 0.4);
            \draw[thick] (-1.2,0) -- (-1.2,-2);
            \draw[thick] (1.2,0) -- (1.2,-2);
            \node at (0, -2.5) {radius $r = 5$};
            \node[right] at (1.3,-1) {height $h = 10$};
        \end{tikzpicture}
        \caption{ Spherical Cylinder with radius $r=5$m and height $h=10$ m}
        \label{fig:fig1}
    \end{figure}
\end{frame}
%%%%%%%%%%%%%%%%%%%%%%%
\begin{frame}{Other Plots}
    \begin{figure}
    \centering
    \includegraphics[width=0.5\linewidth]{images/seminole.png}
    \caption{ Florida State Seminole (Photo credit: \href{https://en.wikipedia.org/wiki/Florida_State_Seminoles}{Wikipedia})}
    \label{fig:enter-label}
    \end{figure}
\end{frame}
%%%%%%%%%%%%%%%%%%%%%%%%%%%%%%%%%%%%%%
\section{Reference}
\begin{frame}{Bibliography}
\bibliographystyle{plainnat}\bibliography{ref}
\end{frame}
\end{document}